\documentclass[11pt]{article}

\usepackage[letterpaper]{geometry}

\usepackage{latexsym}
\usepackage{fullpage}
\usepackage{titlesec}
\usepackage{marvosym}
\usepackage[usenames,dvipsnames]{color}
\usepackage{verbatim}
\usepackage{enumitem}
\usepackage[pdftex]{hyperref}
\usepackage{fancyhdr}
\usepackage{etaremune}
\usepackage{longtable}
\usepackage{fancyhdr}


\pagestyle{fancy}
\fancyhf{}
\fancyfoot{}
\renewcommand{\headrulewidth}{0pt}
\renewcommand{\footrulewidth}{0pt}

% Adjust margins
\addtolength{\oddsidemargin}{-0.5in}
\addtolength{\evensidemargin}{-0.5in}
\addtolength{\textwidth}{1in}
\addtolength{\topmargin}{-.5in}
\addtolength{\textheight}{1.0in}

\urlstyle{same}

\raggedbottom
\raggedright
\setlength{\tabcolsep}{0in}

% Sections formatting
\newcommand{\resumeItem}[2]{
  \item\small{
    \textbf{#1}{: #2 \vspace{-2pt}}
  }
}

\newcommand{\resumeSubheading}[4]{
  \vspace{-1pt}\item
    \begin{tabular*}{0.97\textwidth}{l@{\extracolsep{\fill}}r}
      \textbf{#1} & #2 \\
      \textit{\small#3} & \textit{\small #4} \\The Ohio State
    \end{tabular*}\vspace{-5pt}
}

\newcommand{\resumeSubItem}[2]{\resumeItem{#1}{#2}\vspace{-4pt}}
\newcommand{\resumeSubHeadingListEnd}{\end{itemize}}
\newcommand{\resumeItemListStart}{\begin{itemize}}
\newcommand{\resumeItemListEnd}{\end{itemize}\vspace{-5pt}}


%-------------------------------------------
%%%%%%  CV STARTS HERE  %%%%%%%%%%%%%%%%%%%%%%%%%%%%



\begin{document}
\pagestyle{plain}
\lhead{Kong Jan. 2020}

\begin{center}
\huge \textbf{Sungsik Kong}\\
\vspace{0.2cm}
\large Department of Evolution, Ecology and Organismal Biology\\
\large The Ohio State University\\
\large 400 Aronoff Laboratory\\
\small \href{http://minglabaaa.github.io}http://minglabaaa.github.io/\\
\small \href{mailto:sungsik.kong@gmail.com}sungsik.kong@gmail.com
\end{center}




\hspace{0pt}




\section*{Education}

\begin{longtable}{p{0.15\textwidth}  p{0.1\textwidth} p{0.52\textwidth} p{0.23\textwidth}}
2018--Present & \textbf{Ph.D.} & Evolution, Ecology, and Organismal Biology (EEOB) & The Ohio State University\\
 & & Presidential Fellowship Awardee\\
 & & Advisor: Dr. Laura Kubatko\\
2013--2015 & \textbf{M.Sc.} & Ecology and Evolutionary Biology (EEB) & University of Toronto\\
 & & Advisor: Dr. Robert Murphy\\
2006--2012 & \textbf{B.Sc.} & Ecology and Evolutionary Biology and Zoology & University of Toronto\\
 & & Advisors: Drs. Deborah McLennan and Robert Murphy
\end{longtable}





\section*{Peer-reviewed Journal Articles}

%\subsection*{}
$^\ddag$Co-correspondence; $^\dag$Equal contribution
\begin{etaremune}

\item Glon MG, Broe MB, Crandall KA, Daly M, \textbf{Kong S}, Thoma RF, and Freudenstein JV. (2022) Anchored hybrid enrichment resolves the \textit{Lacunicambarus} (Decapoda: Cambaridae) phylogeny. \textit{Journal of Crustacean Biology} doi: \href{https://doi.org/10.1093/jcbiol/ruab073}10.1093/jcbiol/ruab073

\item \textbf{Kong S} and Kubatko LS. (2021) Comparative performance of popular methods for hybrid detection using genomic data. \textit {Systematic Biology} 70:5 891–907 doi: \href{10.1093/sysbio/syaa092}10.1093/sysbio/syaa092

\item  Sánchez-Pacheco SJ{$^\ddag$}, \textbf{Kong S}{$^\ddag$}, Pulido-Santacruz P, Murphy RW, and Kubatko L. (2020) Median-joining network analysis of SARS-CoV-2 genomes is neither phylogenetic nor evolutionary. \textit{Proc. Natl. Acad. Sci. U.S.A} 117:23 12518–12519 doi: \href{https://doi.org/10.1073/pnas.2007062117}10.1073/pnas.2007062117
	

	
\item Kim M, Nguyen HQ, Yi Y, Ahn J, Kim YE, Jang S, Kim S, \textbf{Kong S}, and Borzée A. (2020) Policy recommendation on whaling, trade and watching of cetaceans in the Republic of Korea. \textit{Biodiversity Journal} 11: 255–258 doi: \href{https://doi.org/10.31396/Biodiv.Jour.2020.11.1.255.258}https://doi.org/10.31396/Biodiv.Jour.2020.11.1.255.258
\item Weiringa JG, Boot MR,  Dantas-Queiroz MV, Duckett D, Fonseca EM, Glon H, Hamilton N, \textbf{Kong S}, Lanna FM, Mattingly KZ, Parsons DJ, Smith ML, Stone BW, Thompson C, Zuo L, and Carstens BC. (2020) Does habitat stability structure intraspecific genetic diversity? It’s complicated… \textit{Frontiers in Biogeography} e45377 doi:\href{https://doi.org/10.21425/F5FBG45377}10.21425/F5FBG45377
\item Borzée A, Purevdorj Z, Kim YI, \textbf{Kong S}, Choe M, Yi Y, Kim K, Kim A, and Jang Y. (2020) Breeding preferences in the Treefrogs \textit{Dryophytes japonicus} (Hylidae) in Mongolia, the northern limit of their range. \textit{Journal of Natural History} 53: 2685-2698 doi:\href{10.1080/00222933.2019.1704458}10.1080/00222933.2019.1704458
\item Groffen J, \textbf{Kong S}, Jang Y, and Borzée A. (2019) The invasive American bullfrog (\textit{Lithobates catesbeianus}) in the Republic of Korea: history and recommendation for population control. \textit{Management of Biological Invasions} 10:3 517–535 doi: \href{10.3391/mbi.2019.10.3.08}10.3391/mbi.2019.10.3.08
\item Bae Y, \textbf{Kong S}, Yi Y, Jang Y, and Borzée A. (2019) Additional new type of threat to \textit{Hynobius} salamander eggs: predation by loaches (\textit{Misgurnus} sp.) in agricultural wetlands. \textit{Animal Biology} 69:4 451–461 doi: \href{https://doi.org/10.1163/15707563-20191070}10.1163/15707563-20191070
\item Borzée A,\textbf{ Kong S}, Didinger C, Nguyen HQ, and Jang Y. (2018) A ring-species or a ring of species? Phylogenetic relationship between two treefrog species around the Yellow Sea: \textit{Dryophytes suweonensis }and \textit{D. immaculatus}. \textit{The Herpetological Journal} 28: 160–170 
\item Kim K, \textbf{Kong S}, Kim YI, Borzée A, Bae Y, and Jang Y. (2018) Japanese hard ticks (\textit{Ixodes nipponensis}) parasitizing on the endangered leopard cat (\textit{Prionaliurus bengalensis euptilura}) in the Republic of Korea. \textit{Animal Systematics, Evolution and Diversity} 34:1 23–26 doi: \href{10.5635/ASED.2018.34.1.032}10.5635/ASED.2018.34.1.032
\item \textbf{Kong S}{$^\dag$}, Sánchez-Pacheco S{$^\dag$}, and Murphy RW. (2016) On the use of Median-Joining Networks in evolutionary biology. \textit{Cladistics} 32: 691–699 doi: \href{https://doi.org/10.1111/cla.12147}10.1111/cla.12147

\end{etaremune}













\section*{Additional Academic Experiences}

\subsection*{Academic Service}
\begin{longtable}{p{0.15\textwidth}  p{0.85\textwidth}}
2019--Present &	\textbf{Journal reviewer for}: \textit{Systematic Biology}; \textit{Genome Biology and Evolution}; \textit{Frontiers of Biogeography}; \textit{Infection, Genetics and Evolution}; \textit{PeerJ};\\
2015--Present &  \textbf{Society Memberships}: Society of Systematic Biologists (SSB), Society for Study of Evolution\\

2020--2021 &  \textbf{Graduate Studies Committee} Department of EEOB, The Ohio State University, OH., USA\\
2019--2020 &	\textbf{Admissions Committee} Department of EEOB, The Ohio State University, OH., USA\\
2016--2017 &	\textbf{Communication Officer} Society for Conservation Biology (SCB), Korean Chapter\vspace{5pt}\\
\end{longtable}





\subsection*{Conference/Workshop Participation}
\begin{longtable}{p{0.15\textwidth}  p{0.85\textwidth}}

%2022 ISCB Madison
%2022 Evolution Cleveland
%2022 Phylogenetic Network symposium organizer Evolution Cleveland

2021 & iEvoBio, Evolution Meeting, virtual\\
2021 & \textbf{Chair/Moderator} in multiple Faux-Live Sessions, Evolution Meeting, virtual\\

2021 & Workshop in Mathematical and Computational Biology, virtual\\
2020 & Future of Systematics Workshop, SSB Meeting, funded by NSF, hosted by Dr. Emily Sessa, University of Florida, Gainesville, FL., USA\\
2020 & SSB Standalone Meeting, Gainesville, FL., USA\\
2019 & Evolution Meeting, Providence, RI., USA\\
2019 & Midwest Phylogenetics Workshop, Itasca Biological Station, funded by NSF, hosted by Dr. Emma Goldberg, University of Minnesota, MN., USA\\
2017 & SCB Asia Section Workshop on Conservation policy, funded by SCB, Bangkok, Thailand\\
2016 & World Congress of Herpetology, Hangzhou and Tonglu, China\\
2015 & Phylogenomics Symposium and Software School, funded by NSF, hosted by Dr. Tandy Warnow, University of Illinois at Urbana-Champaign and University of Michigan, Ann Arbor, MI., USA\\
2015 & SSB Standalone Meeting, University of Michigan, Ann Arbor, MI., USA\vspace{5pt}\\
\end{longtable}


\subsection*{Diversity and Inclusion Activities}
\begin{longtable}{p{0.15\textwidth}  p{0.85\textwidth}}
2021& \textbf{Moderator} International Mixer, Evolution Meeting, virtual\\
2021& \textbf{Moderator} Decolonizing Evolution Workshop, Evolution Meeting, virtual\\
2020 &	\textbf{Moderator} Strategies for Responding to Harassment and Bullying: Improving Workplace Climate, Evolution Community Resources for Early Career Researchers 2020 (ECR$^2$), virtual\\
2020 &	\textbf{Moderator} Fireside ECR Chat, ECR$^2$, virtual\\
\end{longtable}

\clearpage

\section*{Background}

My name is Sungsik (Kevin) Kong and I am a fourth year Ph.D. student at the Ohio State University advised by Dr. Laura Kubatko. My current work involves the application of techniques from statistics to problems in phylogenetic network inference through the development of the computer programs. I have conferred B.Sc. and M.Sc. at the University of Toronto, Canada, advised by Drs. Deborah McLennan and Robert Murphy where I studied behavioral evolution of primates, taxonomy of Asian Pit Vipers, and phylogenetic methods. After I complete my M.Sc. degree, I have backpacked in Asia for a year volunteering in several wildlife sanctuaries, and worked for a year as a researcher at Ewha Womans University in South Korea. The Society of Systematic Biologists (SSB) has been my home academic society throughout my graduate studies and I have received supports from the society in various ways including a publication in \textit{Systematic Biology}, a financial support through \textit{Ad Hoc} fund, and a tremendous amount of encouragements from the members of society. I would like to serve on the SSB Council as the graduate student representative to enhance student participation to the society and service our student members. I believe my experiences in academic service will be extremely helpful to achieve this goal. Moreover, I believe I can be helpful to promote international students' engagement with experiences in diversity and inclusion activities as well as my life experiences as a bilingual, a traveller, an immigrant, and an international student.





\end{document}